
\heading{湖南大学本科生毕业论文(设计)}
\LieOnBadge{本科生毕业论文(设计)}
% 标题全称、分行标题(如果首行够长则可忽略第二个)、分行标题二。
\title{川菜和湘菜在辣度上的对比研究}
\titleOnTheFirstLine{川菜和湘菜在辣度上}
\titleOnTheSecondLine{的对比研究}
\etitle{A Research Conducted on the Spiciness between SiChuan Cusine and Hunan Cusine}
% 作者以及导师
\author{十三香}
\studentid{201212313231}
\major{食品工程}
\faculty{后勤保障部}
\teacher{王守义}

% 声明书和授权内容。
%% 概要与关键词
% 这里填入你自己的内容。
\cabstract{
    摘要是论文内容的简要陈述,是一篇具有独立性和完整性的短文。
    摘要应包括本论文的创造性成果及其理论与实际意义。
    摘要中不宜使用公式、图表,不标注引用文献编号。避免将摘要写成目录式的内容介绍。
% 一定要有空行隔开,否则可能会缩成一坨。
    比如上面是一段,下面是第二段。

}
\eabstract{
    An abstract is a brief summary of a research article, 
    thesis, review, conference proceeding or any in-depth analysis of 
    a particular subject and is often used to help the reader quickly 
    ascertain the paper's purpose. When used, an abstract always 
    appears at the beginning of a manuscript or typescript, 
    acting as the point-of-entry for any given academic paper or patent application. 
    Abstracting and indexing services for various academic disciplines are aimed at 
    compiling a body of literature for that particular subject.
% There must be a blank setting right here to avoid the messing up of line-skip.
    As you can see, this paragraph has the same style as the previous one.

}
\ckeywords{关键词1;~~关键词2;~~关键词3;~~关键词4}
\ekeywords{Key Word1;~~Key Word 2;~~Key Word 3;~~Key Word 4}
%% 概要与关键词结束
%% 如果有需求可以加下面两行以在附录生成缩略表
% \glssetwidest{ADC}
% \makeglossaries
\clearpage