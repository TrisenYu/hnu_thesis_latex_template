
\heading{湖南大学本科生毕业论文(设计)}
\thesistype{本科生毕业论文(设计)}
% 标题全称、分行标题(如果首行够长则可忽略第二个)、分行标题二。
\title{}
\titleOnTheFirstLine{}
\titleOnTheSecondLine{}
\etitle{}
% 作者以及导师
\author{}
\studentid{}
\subject{}
\faculty{}
\teacher{}
% 声明书和授权内容。
\declaredtitle{毕业论文(设计)原创性声明}
\declaration {
    本人郑重声明:所呈交的论文(设计)是本人在导师的指导下独立进行研究所取得的研究成果。
    除了文中特别加以标注引用的内容外,本论文(设计)不包含任何其他个人或集体已经发表或撰写的成果作品。
    对本文的研究做出重要贡献的个人和集体,均已在文中以明确方式标明。本人完全意识到本声明的法律后果由本人承担。
}
\authorizedtitle{毕业论文(设计)版权使用授权书}
\authorization {
    本毕业论文(设计)作者完全了解学校有关保留、使用论文(设计)的规定,
    同意学校保留并向国家有关部门或机构送交论文(设计)的复印件和电子版,
    允许论文(设计)被查阅和借阅。本人授权湖南大学可以将本论文(设计)的全部或部分内容编入有关数据库进行检索,
    可以采用影印、缩印或扫描等复制手段保存和汇编本论文(设计)。
}
\studentsign{学生签名:}
\teachersign{导师签名:}
\cdatename{日期:}
\date{\the\year 年\the\month 月 \the\day 日}


%% 概要与关键词
% 这里填入你自己的内容。
\cabstract{
    摘要是论文内容的简要陈述,是一篇具有独立性和完整性的短文。
    摘要应包括本论文的创造性成果及其理论与实际意义。
    摘要中不宜使用公式、图表,不标注引用文献编号。避免将摘要写成目录式的内容介绍。
}
\eabstract{
    An abstract is a brief summary of a research article, 
    thesis, review, conference proceeding or any in-depth analysis of 
    a particular subject and is often used to help the reader quickly 
    ascertain the paper's purpose. When used, an abstract always 
    appears at the beginning of a manuscript or typescript, 
    acting as the point-of-entry for any given academic paper or patent application. 
    Abstracting and indexing services for various academic disciplines are aimed at 
    compiling a body of literature for that particular subject.
}
\ckeywords{关键词1;~~关键词2;~~关键词3;~~关键词4}
\ekeywords{Key Word1;~~Key Word 2;~~Key Word 3;~~Key Word 4}
%% 概要与关键词结束
\clearpage