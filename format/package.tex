%%%% contact via <kisfg@hotmail.com, haikureimu@hnu.edu.cn>
\usepackage{graphicx}             
\usepackage{subcaption}
\usepackage{caption}
\usepackage[
    figuresright
] {rotating}                % 用于旋转某些东西,如长表格。
\usepackage{geometry}
% includeheadfoot,
% driver=dvipdfmx
% 调节页边距。
\usepackage{tikz-cd}        % 交换图和 tikzpicture 手动绘图的需求。
\usepackage{pgf-pie}
\usepackage{tikz}
\usetikzlibrary{
    decorations.pathmorphing,
    decorations.markings,
    decorations.text,
    datavisualization,
    shapes,
    shadows,
    calc,
    arrows, 
    fit,
    chains, 
    math,
    automata,
    intersections, 
    positioning, 
    bending,
}

%% 这里好像可以直接设的
% 上边距:30mm;下边距:25mm;左边距:30mm;右边距:20mm;行间距为1.5倍行距。
% 上边距 3cm
% 下边距 2.5cm
% 页眉到上边缘的距离 1.5cm
% 页脚到下边缘的距离 1.75cm
\geometry{
    % includeheadfoot,
    a4paper,
    % paperwidth=21cm,
    % paperheight=29.7cm,
    % height = textheight,
    left=30mm,
    right=20mm,
    top=30mm,
    bottom=25mm,
    footskip=7.5mm,
    headsep=10mm
    % 1.5cm + 9pt
    % 1.8177cm - 21pt
    % 1.79942
    % 2.37128
}
% \setlength{\headsep}{1.5cm+1in}
% \setlength{\footskip}{0.75cm+1in}
\usepackage{enumerate}      % 罗列编号用的宏包
\usepackage{titlesec}       % 控制标题的宏包
\usepackage{titletoc}       % 控制目录的宏包
\usepackage{fancyhdr}       % fancyhdr宏包 支持页眉和页脚的相关定义
\usepackage[UTF8]{ctex}     % 支持中文显示
\usepackage{xcolor}         % 支持彩色
\usepackage[
    scr=boondox,
    cal=esstix
] {mathalpha}               % mathcal 正确显示
\usepackage{
    amsmath, 
    amsthm, 
    amssymb, 
    amsbsy,
    txfonts,                % 课本里头用的公式字体类型
    % mathptmx,             % 和 txfonts 是一家的。读者应按需使用。
    amsfonts,
    times
}                           % 数学宏包
\usepackage[utf8]{inputenc}
\usepackage{bm}
\usepackage[
    below               
] {placeins}                
\usepackage{flafter}        % 使得所有浮动体不能被放置在其浮动环境之前,以免浮动体在引述它的文本之前出现.
\usepackage{multirow}       % 使用Multirow宏包,使得表格可以合并多个row格
\usepackage{booktabs}       % 表格,横的粗线;\specialrule{1pt}{0pt}{0pt}
\usepackage{longtable}      % 支持跨页的表格。
\usepackage{diagbox}        % 斜线
\usepackage{tabularx}       % 自动设置表格的列宽
\usepackage{setspace}       % space 环境
\usepackage{booktabs}
\usepackage{float}          % 浮动体控制
\usepackage{balance}        % 自动调整超出margin的文字
\usepackage{enumitem}       % 使用enumitem宏包,改变列表项的格式
\usepackage{calc}           % 长度可以用+ - * / 进行计算
% \usepackage[
%     amsmath, 
%     thmmarks, 
%     hyperref
% ] {ntheorem}            
% 定理类环境宏包,其中 amsmath 选项用来兼容 AMS LaTeX 的宏包
\usepackage{indentfirst}    % 首行缩进宏包
\usepackage{CJKutf8}        % 用在UTF8编码环境下,它可以自动调用CJK,同时针对UTF8编码作了设置
\usepackage{CJK}
\usepackage{fancyhdr}       % 页眉页脚
\usepackage{lastpage}
\usepackage{layout}
\usepackage{overpic}
\usepackage[
    titles                  % 保证目录换行
] {tocloft}
\usepackage[
    backend=biber,          % 这个处理后端才能显示文献类型符如 [OL]
    style=gb7714-2015, 
    maxcitenames=3,         % 最多输出三个。
    mincitenames=1,
    maxbibnames=3,
    uniquelist=false,
    uniquename=false,
    doi=false,              % doi 不要了,文献一多就难调。
    url=false,              % 某些时候要用这个少打印参考文献内的链接
    % 12 页:mirrors.sjtug.sjtu.edu.cn/ctan/macros/latex/contrib/biblatex-contrib/biblatex-gb7714-2015/biblatex-gb7714-2015.pdf
    gbnamefmt=lowercase,    % 外国人名小写
    gbnamefmt=fullname,     % 名前姓后,全名
    gbpub=false,
    gbalign=gb7714-2015,    % 良好的左对齐
    gbnoauthor=true,        % 佚名作者用
    gbtype=true,            % 显式说明使用文献类型
    gbpunctin=false,        % 去双斜杠
] {biblatex}
% \usepackage{url}
\usepackage{ifthen}         % 条件编译可能会用得到
\usepackage{threeparttable} % 编制复杂表格
\usepackage[
    version=4
] {mhchem}                  % 化学式宏包
\usepackage{listings}       % 代码高亮宏包
\usepackage[T1]{fontenc}
\usepackage[
    figure, 
    table
] {totalcount}              % 统计图表是否存在。src: tex.stackexchange.com/a/297657 
\usepackage{ 
    scrextend, 
    scalerel,
    makecell
}
\usepackage{microtype}      % 不要拿掉,控制连字的关闭。
\usepackage{tabularray}
\usepackage[
    bottom,                 % 固定脚注
    flushmargin,            % 靠左。
    hang
] {footmisc}

\usepackage{
    pgfplots,
    pgfplotstable           % 数据分析绘制图表
}

\usepackage{array}
\usepackage{adjustbox}
\usepackage{svg}            % 矢量包需要自己配inkscape环境!
% https://blog.csdn.net/ya6543/article/details/113075774
% \usepackage{nomencl}        % 术语表
% 设置 tikz 插图的字体大小。
\tikzset{
    every picture/.style={font issue=\footnotesize},
    font issue/.style={execute at begin picture={#1\selectfont}}
}

% 最后再加载这几个。
\usepackage[
    pdftex,
    unicode,
    plainpages=false,
    pdfpagelabels,
    % pdfstartview=FitH,
    bookmarks,
    bookmarksnumbered=true,
    bookmarksopen=true,
    colorlinks=true,
    pdfborder={0 0 1},
    % 全部放黑。
    citecolor=black,
    linkcolor=black,
    anchorcolor=black,
    urlcolor=black,
    breaklinks=true
] {hyperref}

\usepackage[
    style=super, 
    nogroupskip, % 组间不设置间隔。
    numberline,
    xindy,
    section,
    esclocations,
    % counter=part
] {glossaries}
% https://tex.stackexchange.com/a/97179
\usepackage[
    automake,
    acronym,
    postdot
] {glossaries-extra}


\setlength{\glsdescwidth}{0.8\hsize}
% 符号表
\newglossarystyle{CustomSymbolTableStyle}{
    %设置说明列宽度:
    \setlength{\glsdescwidth}{0.6\linewidth}
    %设置页码列宽度:
    \setlength{\glspagelistwidth}{0.1\linewidth}
    \renewenvironment{theglossary} {
        \begin{longtable}[c]{@{}cp{\glsdescwidth}m{\glspagelistwidth}<{\centering}@{}}
            % 这里填的就都是缩略项。
    } {
        \specialrule{1pt}{0pt}{0pt}
        \end{longtable}
    }
  
    %设置没有表头, 以及内容
    \renewcommand*{\glossaryheader}{
        \caption{本文所用缩略术语汇总}
        \label{customGlossaries}\\
        \specialrule{1pt}{0pt}{0pt}
        缩略名 &\multicolumn{1}{c}{全称及中文译名} &首次出现页码 \\
        \hline
        \endfirsthead
        \specialrule{1pt}{0pt}{0pt}
        缩略名 &\multicolumn{1}{c}{全称及中文译名} &首次出现页码 \\
        \hline
        \endhead
    }
    % 设置分组间没有表头:
    \renewcommand*{\glsgroupheading}[1]{}%
    % 主条目第一列名称,第二列说明,第三列页码:
    \renewcommand{\glossentry}[2]{
        \glsentryitem{##1}
        \glstarget{##1} {\glossentryname{##1}} &\glossentrydesc{##1}. & ##2\\
    }
    % 子条目第一列空置,第二列说明,第三列页码:
    \renewcommand*{\subglossentry}[3]{
        &\glssubentryitem{##2}\glstarget{##2}{\strut}\glossentrydesc{##2}. &##3\\
    }%
    % 定义分组间空白:
    \renewcommand*{\glsgroupskip} {
        \ifglsnogroupskip
        \else & &\\
        \fi
    }
}
\lstdefinestyle{DOS}
{
    backgroundcolor=\color{white},
    basicstyle=\normalsize\color{black}\rmfamily,
    columns=fullflexible,
}
\usepgfplotslibrary{groupplots}
\pgfplotsset{width=7cm,compat=1.18}

\usepackage{ipaex-type1}    % 日语字体包