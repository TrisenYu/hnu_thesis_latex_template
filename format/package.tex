%%%% contact via <kisfg@hotmail.com, haikureimu@hnu.edu.cn>
\usepackage{graphicx}             
\usepackage{subcaption}
\usepackage{caption}
\usepackage[
    figuresright
] {rotating}                % 用于旋转某些东西,如长表格
\usepackage[
    includeheadfoot,
    % driver=dvipdfmx
] {geometry}                % 调节页边距
\usepackage{tikz-cd}        % 交换图和 tikzpicture 手动绘图的需求
\usepackage{tikz}
\usetikzlibrary {
    decorations.pathmorphing,
    decorations.markings,
    shapes, 
    calc,
    arrows, 
    chains, 
    math,
    automata,
    intersections, 
    positioning, 
    bending,
    arrows.meta
}

%%% 页边距定义
% 上边距:30mm;下边距:25mm;左边距:30mm;右边距:20mm;行间距为1.5倍行距。
% 上边距 3cm
% 下边距 2.5cm
% 页眉到上边缘的距离 1.5cm
% 页脚到下边缘的距离 1.75cm,但是实际看有点远,改用1.8
\geometry{
    left=3cm,
    right=2cm,
    top=1.5cm,
    bottom=1.8cm
}
\usepackage{enumerate}      % 罗列编号用的宏包
\usepackage{titlesec}       % 控制标题的宏包
\usepackage{titletoc}       % 控制目录的宏包
\usepackage{fancyhdr}       % fancyhdr宏包 支持页眉和页脚的相关定义
\usepackage[UTF8]{ctex}     % 支持中文显示
\usepackage{xcolor}         % 支持彩色
\usepackage[
    scr=boondox,
    cal=esstix
] {mathalpha}               % mathcal 正确显示
\usepackage{
    amsmath, 
    amsthm, 
    amssymb, 
    amsbsy,
    txfonts,                % 课本里头用的公式字体类型
    % mathptmx,             % 和 txfonts 是一家的。读者应按需使用。
    amsfonts,
    times
}                           % 数学宏包
\usepackage[utf8]{inputenc}
\usepackage{bm}
\usepackage[
    below               
] {placeins}                
\usepackage{flafter}        % 使得所有浮动体不能被放置在其浮动环境之前,以免浮动体在引述它的文本之前出现.
\usepackage{multirow}       % 使用Multirow宏包,使得表格可以合并多个row格
\usepackage{booktabs}       % 表格,横的粗线;\specialrule{1pt}{0pt}{0pt}
\usepackage{longtable}      % 支持跨页的表格。
\usepackage{tabularx}       % 自动设置表格的列宽
\usepackage{setspace}
\usepackage{float}          % 浮动体控制
\usepackage{balance}        % 自动调整超出margin的文字
\usepackage{enumitem}       % 使用enumitem宏包,改变列表项的格式
\usepackage{calc}           % 长度可以用+ - * / 进行计算

% \usepackage[
%     amsmath, 
%     thmmarks, 
%     hyperref
% ] {ntheorem}                % 定理类环境宏包,其中 amsmath 选项用来兼容 AMS LaTeX 的宏包
\usepackage{indentfirst}    % 首行缩进宏包
%% linux 下 xeLatex 不可以用 CJK 和 CJKutf8。
%% 取消掉了好像也对windows没什么影响。
\usepackage{fancyhdr}       % 页眉页脚
\usepackage{lastpage}
\usepackage{layout}
\usepackage{overpic}
\usepackage[
    titles                  % 保证目录换行
] {tocloft}
\usepackage[
    backend=biber,          % 这个处理后端才能显示文献类型符如 [OL]
    style=gb7714-2015, 
    maxcitenames=3,         % 最多输出三个。
    mincitenames=1,
    maxbibnames=1,
    minbibnames=1,          % 某些时候,参考文献中最多只有 1 个作者
    uniquelist=false,
    uniquename=false,
    doi=false,              % doi 有时可能也不要了。
    url=false,              % 某些时候要用这个少打印参考文献内的链接
    % 12 页:mirrors.sjtug.sjtu.edu.cn/ctan/macros/latex/contrib/biblatex-contrib/biblatex-gb7714-2015/biblatex-gb7714-2015.pdf
    gbnamefmt=lowercase,    % 外国人名小写
    gbnamefmt=givenahead,   % 名前姓后
    gbpub=false,
    gbalign=gb7714-2015,    % 良好的左对齐
    gbnoauthor=true,        % 佚名作者用
    gbtype=true,            % 显式说明使用文献类型
    gbpunctin=false
] {biblatex}
\usepackage[
    % dvipdfmx,
    unicode,
    pdfstartview=FitH,
    bookmarks,
    bookmarksnumbered=true,
    bookmarksopen=true,
    colorlinks=true,
    pdfborder={0 0 1},
    % 全部放黑。
    citecolor=black,
    linkcolor=black,
    anchorcolor=black,
    urlcolor=black,
    breaklinks=true
] {hyperref}
\usepackage{url}
\usepackage{ifthen}         % 条件编译可能会用得到
\usepackage{threeparttable} % 编制复杂表格
\usepackage[
    version=4
] {mhchem}                  % 化学式宏包
\usepackage{listings}       % 代码高亮宏包
\usepackage[T1]{fontenc}
\usepackage[
    figure, 
    table
] {totalcount}              % 统计图表是否存在。src: tex.stackexchange.com/a/297657 
\usepackage{ 
    scrextend, 
    scalerel,
    makecell
}
\usepackage{microtype}      % 不要拿掉,控制连字的关闭
\usepackage{tabularray}
\usepackage{moreverb}
\usepackage{keyval}
\usepackage{latexsym}
% \usepackage{gnuplottex}
%% https://stackoverflow.com/a/3527521
\usepackage[
    bottom                  % 固定脚注于页面底部
]{footmisc}
\usepackage{
    pgfplots,
    % pgfplotstable           % 数据分析绘制图表
    siunitx                 % 国际单位制
}
