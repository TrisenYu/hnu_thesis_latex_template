% !Mode:: "TeX:UTF-8"

\usepackage{graphicx}                       % 支持插图处理
\usepackage[includeheadfoot]{geometry}      % 调节页边距。
%     left=3cm,
%     right=2cm,
\geometry{
    left=3cm,
    right=2cm
}
\usepackage{float}
                                            % 支持版面尺寸设置
\usepackage{titlesec}                       % 控制标题的宏包
\usepackage{titletoc}                       % 控制目录的宏包
\usepackage{fancyhdr}                       % fancyhdr宏包 支持页眉和页脚的相关定义
\usepackage[UTF8]{ctex}                     % 支持中文显示
\usepackage{color}                          % 支持彩色
\usepackage{amsmath}                        % AMSLaTeX宏包 用来排出更加漂亮的公式
\usepackage{amssymb}                        % 数学符号生成命令
\usepackage[below]{placeins}                % 允许上一个section的浮动图形出现在下一个section的开始部分,还提供\FloatBarrier命令,使所有未处理的浮动图形立即被处理
\usepackage{flafter}                        % 使得所有浮动体不能被放置在其浮动环境之前,以免浮动体在引述它的文本之前出现.
\usepackage{multirow}                       % 使用Multirow宏包,使得表格可以合并多个row格
\usepackage{booktabs}                       % 表格,横的粗线;\specialrule{1pt}{0pt}{0pt}
\usepackage{longtable}                      % 支持跨页的表格。
\usepackage{tabularx}                       % 自动设置表格的列宽
\usepackage{setspace}
\usepackage{subfigure}                      % 支持子图 %centerlast 设置最后一行是否居中
\usepackage[subfigure]{ccaption}            % 支持子图的中文标题
\usepackage{float}                          % 浮动体控制
\usepackage{balance}
\usepackage{enumitem}                       % 使用enumitem宏包,改变列表项的格式
\usepackage{calc}                           % 长度可以用+ - * / 进行计算
\usepackage{txfonts}                        % 字体宏包
\usepackage{bm}                             % 处理数学公式中的黑斜体的宏包
\usepackage[amsmath,thmmarks,hyperref]{ntheorem}  % 定理类环境宏包,其中 amsmath 选项用来兼容 AMS LaTeX 的宏包
\usepackage{CJKnumb}                        % 提供将阿拉伯数字转换成中文数字的命令
\usepackage{indentfirst}                    % 首行缩进宏包
\usepackage{CJKutf8}                        % 用在UTF8编码环境下,它可以自动调用CJK,同时针对UTF8编码作了设置。
\usepackage{CJK}
\usepackage{fancyhdr}
\usepackage{lastpage}
\usepackage{layout}
\usepackage{overpic}
\usepackage[titles,subfigure]{tocloft}                   
\usepackage[
    backend=biber,       % 这个处理后端才能显示文献类型符如 [OL]
    style=gb7714-2015, 
    gbnamefmt=lowercase, % 外国人名小写 
    gbpub=false,
    gbalign=gb7714-2015, % 良好的左对齐
    gbnoauthor=true,     % 佚名作者用
    gbtype=true,         % 显式说明使用文献类型。
    gbpunctin=false
] {biblatex}
\usepackage[
    dvipdfm, unicode,
    pdfstartview=FitH,
    bookmarksnumbered=true,
    bookmarksopen=true,
    colorlinks=true,
    pdfborder={0 0 1},
    citecolor=black,
    linkcolor=black,
    anchorcolor=black,
    urlcolor=black,
    breaklinks=true
]{hyperref}
% \usepackage{minted}
% \usemintedstyle{lovelace}