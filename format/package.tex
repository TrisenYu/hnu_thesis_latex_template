%%%% contact via <kisfg@hotmail.com, haikureimu@hnu.edu.cn>
% 需要传入dvipfmx 参数使插图正常工作。
\usepackage{graphicx}             
\usepackage{subcaption}
\usepackage{caption}
\usepackage[
    figuresright
] {rotating}            % 用于旋转某些东西,如长表格。
\usepackage[
    includeheadfoot,
    driver=dvipdfmx
] {geometry}            % 调节页边距。
\usepackage{tikz-cd}
\usepackage{tikz}
% 左边距 3cm
% 右边距 2cm
% left=3cm,
% right=2cm,
\geometry{
    left=3cm,
    right=2cm
}
\usepackage{enumerate}  % 罗列编号用的宏包
\usepackage{titlesec}   % 控制标题的宏包
\usepackage{titletoc}   % 控制目录的宏包
\usepackage{fancyhdr}   % fancyhdr宏包 支持页眉和页脚的相关定义
\usepackage[UTF8]{ctex} % 支持中文显示
\usepackage{color}      % 支持彩色
\usepackage[
    scr=boondox,
    cal=esstix
] {mathalpha}           % mathcal 正确显示
\usepackage{
    amsmath, 
    amsthm, 
    amssymb, 
    amsbsy,
    % txfonts,          % 课本里头用的公式字体类型
    % mathptmx,         % 和 txfonts 是一家的
    % mathrsfs          % 书法字体包,有额外警报
    amsfonts,
}                       % 数学宏包
\usepackage{bm}
\usepackage[
    below               
] {placeins}                
\usepackage{flafter}    % 使得所有浮动体不能被放置在其浮动环境之前,以免浮动体在引述它的文本之前出现.
\usepackage{multirow}   % 使用Multirow宏包,使得表格可以合并多个row格
\usepackage{booktabs}   % 表格,横的粗线;\specialrule{1pt}{0pt}{0pt}
\usepackage{longtable}  % 支持跨页的表格。
\usepackage{tabularx}   % 自动设置表格的列宽
\usepackage{setspace}
\usepackage{float}      % 浮动体控制
\usepackage{balance}    % 自动调整超出margin的文字
\usepackage{enumitem}   % 使用enumitem宏包,改变列表项的格式
\usepackage{calc}       % 长度可以用+ - * / 进行计算

% \usepackage[
%     amsmath, 
%     thmmarks, 
%     hyperref
% ] {ntheorem}            
% 定理类环境宏包,其中 amsmath 选项用来兼容 AMS LaTeX 的宏包
\usepackage{indentfirst}% 首行缩进宏包
\usepackage{CJKutf8}    % 用在UTF8编码环境下,它可以自动调用CJK,同时针对UTF8编码作了设置
\usepackage{CJK}
\usepackage{fancyhdr}   % 页眉页脚
\usepackage{lastpage}
\usepackage{layout}
\usepackage{overpic}
\usepackage[
    titles              % 保证目录换行
] {tocloft}
\usepackage{times}
\usepackage[
    backend=biber,      % 这个处理后端才能显示文献类型符如 [OL]
    style=gb7714-2015, 
    gbnamefmt=lowercase,% 外国人名小写 
    gbpub=false,
    gbalign=gb7714-2015,% 良好的左对齐
    gbnoauthor=true,    % 佚名作者用
    gbtype=true,        % 显式说明使用文献类型
    gbpunctin=false
] {biblatex}

\usepackage[
    dvipdfmx,
    unicode,
    pdfstartview=FitH,
    bookmarks,
    bookmarksnumbered=true,
    bookmarksopen=true,
    colorlinks=true,
    pdfborder={0 0 1},
    % 全部放黑。
    citecolor=black,
    linkcolor=black,
    anchorcolor=black,
    urlcolor=black,
    breaklinks=true
] {hyperref}
\usepackage{ifthen}     % 条件编译可能会用得到
\usepackage{threeparttable} % 编制复杂表格
\usepackage{tabularray} % 这个没有的话,只能用 tabular 画表而已了
\usepackage[
    version=4
] {mhchem}     % 化学式宏包
