\section{相关工作}
\label{section:relatedWork}
\subsection{eBPF}
扩展的伯克利包过滤器(extended Berkeley Packet Filter,eBPF)是一项Linux内核拓展技术,支持动态地将用户编写的逻辑通过eBPF虚拟机加载使其运行于内核态上\nolinebreak
\cite{sunFindingCorrectnessBugs2024, YIHeCrossContainer}。\nolinebreak
这一直接运行于内核态而无需通过切换用户态和内核态之间的能力,既减少了数据复制次数提高了内核执行性能,又充分扩展了内核的功能\nolinebreak
\cite{YIHeCrossContainer, ZhangZiJunLinuxXiTonge};\nolinebreak
eBPF的这种灵活性被用于多种特定的任务\cite{HaoValidating},如网络包过滤\cite{10.1145/3371038}、安全监控\cite{9110434}以及其它方面\cite{280870, 258973}。\nolinebreak

\subsubsection{eBPF运行流程}
用户首先需要编写包含BPF指令的程序段,通过BPF的系统调用加载BPF程序。程序运行时只能操控11个寄存器(为R0至R10),只能在一个固定大小的栈上支持四种基本运算(加载、存储、算术运算和分支判断)\nolinebreak
\cite{HaoValidating}。\nolinebreak
BPF程序经过加载时会经过编译工具链(如clang)编译为BPF字节码,而后经由BPF系统调用加载到内核态等待通过eBPF验证器的静态验证\nolinebreak
\cite{zhengBpftimeUserspaceEBPF2023}。\nolinebreak
通过验证后,会经由BPF Just-In-Time(JIT)编译器编译为可执行的机器码并将其加载到虚拟机中执行。
\subsubsection{eBPF安全性问题}
仅eBPF的验证器的处理逻辑,就贡献了eBPF本身超过半数的CVE\cite{hive}。
\subsection{Web Assembly}
Web Assembly(wasm)是一种为增强Web浏览器性能和拓展性而设计的专为高级编程语言(如C,C++,rust和Go)的中间编译目标\nolinebreak
\cite{lehmannWasabiFrameworkDynamically2019, lehmannEverythingOldNew, bhansaliFirstLookCode2022, waseemIssuesTheirCauses2024}。\nolinebreak
因其编译器提供了对底层内存布局的控制以及运行时跨硬件指令集的转换,wasm提供了接近原生的执行性能和跨平台的可扩展性\nolinebreak
\cite{lehmannEverythingOldNew, waseemIssuesTheirCauses2024, lehmannWasabiFrameworkDynamically2019, JayProvablySafe}。\nolinebreak
此外,在设计上 wasm 代码被限制在沙盒环境下运行\nolinebreak
\cite{johnsonWaVeVerifiablySecure2023,WasmbpfStreamliningEBPF2024},\nolinebreak
这确保了其运行时安全,并成功地让其充当浏览器之外运行不可信程序的沙盒\cite{narayanSwivelHardeningWebAssembly, WebAssemblySummaryOnSecurity}。

然而 wasm
\subsubsection{wasm运行流程}

%%%% 引用原文
%% lehmannEverythingOldNew
% WebAssembly is an increasingly popular bytecode language 
% that offers a compact and portable representation, fast execution, and a low-level memory model [32]. 
% Announced in 2015 [19] and implemented by all major browsers in 2017 [65], 
% WebAssembly is supported by 92% of all global browser installations as of June 2020.
%
%% lehmannWasabiFrameworkDynamically2019:
% WebAssembly [2, 25] is a new, low-level binary instruction format for the web. 
% Its core use case is as a compilation target for systems programming languages like C, C++, or Rust. 
% By providing low-level control over the memory layout and by closely mapping to hardware instructions, 
%  WebAssembly provides near-native and predictable performance
%
%% johnsonWaVeVerifiablySecure2023
% WebAssembly (Wasm) is a portable bytecode designed to run everywhere at near-native speeds [1], [2]. 
% Unlike most other bytecodes, Wasm was designed with safety in mind from the start: 
% Wasm code runs in a sandboxed environment, because the compiler (or interpreter) inserts 
% runtime checks that restrict the code to its own region of memory.
%
%% 安全综述
% 首先,Wasm 通过在设计中引入强类型系统 [10]、软件故障隔离 [11]、安全控制流 [12]、线性内存 [13] 等多种安全语言特性,
% 保证了程序运行的安全性。其次,Wasm 采用基于栈式虚拟机的抽象指令集,
% 并在设计中兼顾了空间占用与执行效率,使其能够充分利用各种平台上硬件功能,实现了接近原生代码的高执行效率。
%
%% WasmbpfStreamliningEBPF2024
% Its key advantages include platform independence, security through sandboxing, 
% and near-native performance. By compiling programs into a portable binary format, 
% Wasm ensures that they can run consistently across diverse architectures and operating systems 
% without requiring source code modifications.
%
%% Issues and Their Causes in WebAssembly Applications: An Empirical Study
% WebAssembly (Wasm) as a binary instruction format enhances the performance and 
% security of applications in web-based execution  environments [1]. 
% It serves as a potential compilation target for a variety of programming languages 
% including C, C++, and Rust, marking a significant milestone in web development [2].
%
%% Swivel: Hardening WebAssembly against Spectre
% WebAssembly (Wasm) is a portable bytecode originally designed to safely run native code (e.g., C/C++ and Rust) in the browser [27]. 
% Since its initial design, though, Wasm has been increasingly used to sandbox untrusted code outside the browser.
%
%% bhansaliFirstLookCode2022 # A First Look at Code Obfuscation for WebAssembly
% WebAssembly (Wasm) is a compilation target for high-level languages located within browsers 
% that facilitates the creation of highperformance web applications 
% (e.g., games, portable languages, cryptographic computation) that run at near native speed [20].