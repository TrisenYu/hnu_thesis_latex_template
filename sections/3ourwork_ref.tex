\section{内容要求}
\label{section:optimizations}
\subsection{论文题目}
    题目应该简短、明确、有概括性。通过题目,能大致了解论文内容、专业特点和学科范畴。但字数要适当,一般不宜超过20字。必要时可加副标题。
\subsection{摘要与关键词}
    \subsubsection{论文摘要}
        摘要应概括反映出毕业论文(设计)的内容、方法、成果和结论。摘要中一般不宜使用公式、图表,
        不标注引用文献编号。中文摘要以300左右字为宜、外文摘要以250个实词左右为宜。
    \subsubsection{关键词}
        关键词是供检索用的主题词条,应采用能覆盖论文(设计)主要内容的通用技术词条(参照相应的技术术语标准),尽量从《汉语主题词表》中选用,
        未被词表收录的新学科、新技术中的重要术语和地区、人物、文献等名称,也可作为关键词标注。
        关键词一般为3~8个,按词条的外延层次排列(外延大的排在前面)。
        关键词应以与正文不同的字体字号编排在摘要下方。多个关键词之间用分号分隔。中英文关键词应一一对应。
\subsection{目录}
    目录按章、节、条三级标题编写,要求标题层次清晰。目录中的标题要与正文中标题一致。
    目录中应包括绪论、报告(论文)主体、结论、致谢、参考文献、附录等。
\subsection{正文}
    正文是毕业论文(设计)的主体和核心部分,一般应包括绪论、报告(论文)主体及结论等部分。
    \subsubsection{绪论}
        绪论一般作为第一章,是毕业论文(设计)主体的开端。绪论应包括:毕业(设计)的背景及目的;
        国内外研究状况和相关领域中已有的成果;设计和研究方法;设计过程及研究内容等。绪论一般不少于1.5千字。
    \subsubsection{主体}
        主体是毕业论文(设计)的主要部分,应该结构合理、层次清楚、重点突出、文字简练、通顺。主体的内容应包括以下各方面:
        \begin{itemize}
            \item[1.] 毕业论文(设计)总体方案设计与选择的论证。
            \item[2.] 毕业论文(设计)各部分(包括硬件与软件)的设计计算。
            \item[3.] 试验方案设计的可行性、有效性以及试验数据的处理及分析。
            \item[4.] 对本研究内容及成果进行较全面、客观的理论阐述,应着重指出本研究内容中的创新、改进与实际应用之处。
                      理论分析中,应将他人研究成果单独书写,并注明出处,不得将其与本人的理论分析混淆在一起。
                      对于将其他领域的理论、结果引用到本研究领域者,应说明该理论的出处,并论述引用的可行性与有效性。
            \item[5.] 自然科学的论文应推理正确,结论清晰,无科学性错误。
            \item[6.] 管理和人文学科的论文应包括对所研究问题的论述及系统分析、比较研究,模型或方案设计,案例论证或实证分析,
                      模型运行的结果分析或建议、改进措施等。
        \end{itemize}
    \subsection{结论}
        结论是毕业论文(设计)的总结,是整篇设计报告(论文)的归宿。要求精炼、准确地阐述自己的创造性工作或新的见解及其意义和作用,还可进一步提出需要讨论的问题和建议。
    \subsection{参考文献}
        按正文中出现的顺序列出直接引用的主要参考文献。
        毕业论文(设计)的撰写应本着严谨求实的科学态度,凡有引用他人成果之处,均应按论文中所出现的先后次序列于参考文献中。
        并且只列出正文中以标注形式引用或参考的有关著作和论文。{\bfseries\song{一篇论著在论文中多处引用时,在参考文献中只能出现一次,序号以第一次出现的位置为准。}}
    \subsection{致谢}
        致谢中主要感谢导师和对论文工作有直接贡献及帮助的人士和单位。
    \subsection{附录}
        对于一些不宜放入正文中、但作为毕业论文(设计)又是不可缺少的部分,或有重要参考价值的内容,可编入毕业论文(设计)的附录中。
        例如,过长的公式推导、重复性的数据、图表、程序全文及其说明等。


