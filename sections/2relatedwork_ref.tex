\section{相关工作}
\label{section:relatedWork}
\subsection{eBPF}
    eBPF 是一项基于事件驱动的 Linux 内核拓展技术,支持动态地将用户编写的代码通过 eBPF 虚拟机加载到内核态上执行,
    而无需修改内核代码、插入内核模块或重启系统
    \cite{sunFindingCorrectnessBugs2024, YIHeCrossContainer, riceLearningEBPFProgramming2023, }。
    这一直接运行于内核态而无需通过切换用户态和内核态之间的能力,既减少了数据复制次数
    又扩展了内核的功能\cite{YIHeCrossContainer, LinuxEbpfProtection};
    eBPF 的这种灵活性被用于加速特定的任务\cite{HaoValidating},
    如网络包过滤\cite{10.1145/3371038, TCPdump}、网络流量监控\cite{9110434}以及
    文件数据存储\cite{kernelStorage}和 FPGA 接收网络数据流\cite{258973}等。

    \subsubsection{eBPF 运行流程}
        eBPF 的运行流程如图 所示。
        用户首先需要通过相应的 eBPF 框架来编写程序,经过如LLVM/clang的编译工具链
        编译为 eBPF 字节码\cite{riceLearningEBPFProgramming2023}。
        eBPF 字节码再通过 bpf() 加载到内核态\cite{riceLearningEBPFProgramming2023},
        并创建 eBPF 表来维护同一 eBPF 程序的执行状态和 eBPF 程序之间的状态分享
        \cite[2]{bensonNetEditOrchestrationPlatform2024}。
        未经过运行的 eBPF 字节码会由 eBPF 静态验证器来实施检查
        \cite{zhengBpftimeUserspaceEBPF2023, HaoValidating}。
        检查内容包括穷举所有可能的执行情况,eBPF 程序的内存访问情况,程序能否结束,循环次数,无效指令等
        \cite{286467, riceLearningEBPFProgramming2023}。
        验证程序不会对内核造成影响后,最后经由 JIT 编译器编译为可执行的机器码\cite{FuzzOnEBPF},
        并被加载到内核环境下的虚拟机中等待特定事件触发执行\cite{maoMerlinMultitierOptimization2024}。
        此外,eBPF 还可以依靠 Linux 提供的 Kprobe 技术将程序插入到几乎任何内核函数所在的位置
        \cite{riceLearningEBPFProgramming2023}。
    
    \subsubsection{eBPF 字节码指令集}        
        eBPF 程序在运行时只能操控 11 个 64 位寄存器,为 R0 至 R10,其功能分别是:
        \begin{itemize}
            \item[(1)] R0:保存函数返回值和eBPF程序退出值;
            \item[(2)] R1$\sim$R5:保存函数调用参数。如参数少于五个不会使用所有寄存器,超出五个则会以压栈的形式传参;
            \item[(3)] R6$\sim$R9:保存调用函数的上下文;
            \item[(4)] R10: 只读的栈帧寄存器,用于访问栈。
        \end{itemize}
        % eBPF结构体
        同时,eBPF 程序能在一个固定大小的栈上执行四种基本运算:访存、算术、分支判断和调用\cite{gershuniSimplePreciseStatic2019}。


    \subsubsection{eBPF 存在的问题}
        Mohamed Mohamed Husain 等人撰写的文献\parencite{mohamedUnderstandingSecurityLinux2023}
        汇总整理了截至 2023 年 8 月为止的 eBPF 的漏洞报告列表。
        在比重上,eBPF-Helper(eBPF内核帮助函数) 贡献了 16.7\% 的CVE;
        eBPF即时编译器贡献了 5.6\%;eBPF核心贡献了 11.1\%;
        eBPF验证器则占比最大,为 44.4\%,其它则占 22.2\%。
        
        因内核稳定性需要,eBPF 规范不支持浮点运算\cite{10.1145/3674213.3674219},也不允许死循环的存在
        \cite{riceLearningEBPFProgramming2023};还会使用 eBPF 验证器检查 eBPF 程序中是否包含潜在的不安全操作和
        不可接受的性能开销\cite{PKSeBPFIsolation}。
        而 eBPF 的验证器的处理逻辑,截至 2024 年,就产生了 eBPF 本身超过半数的 CVE\cite{hive}。
        而且,当前的 eBPF 在验证 eBPF 程序上有所局限。eBPF 的调用栈最大深度只允许为 512
        \cite[30]{riceLearningEBPFProgramming2023},且经 JIT
        编译后的指令数对于非特权程序最多不能超过 4096 条,特权程序不能超过 100 万条
        \cite{limSafeBPFHardwareassistedDefenseindepth2024};
        eBPF 的验证器存在假阳性误报\cite{hive, gershuniSimplePreciseStatic2019}以及李浩和古金宇等人在文献
        \parencite[2,5,6]{PKSeBPFIsolation}指出的假阴性(即验证器自身的漏洞被利用于恶意程序的绕过)问题。

        另外,对于字节码解释编译阶段,申文博等人的研究\parencite{interpreterhijeck}
        表明 eBPF 字节码缺少注入与劫持防护,其解释器在执行时不会检测所执行的程序,其解释流可以劫持并用于
        执行任意的字节码,可被利用绕过现有的内核代码完整性保护机制,实现内核任意代码注入攻击。


        这些事实,使得开发者不得不从验证器的角度开发\cite{PKSeBPFIsolation}或重构\cite{HaoValidating}代码,
        加重了开发者的负担。

\subsection{Web Assembly 概况}
    wasm 是一种为增强Web浏览器性能和拓展性而面向 C,C++,rust 和 Go 等高级编程语言设计的
    紧凑的\cite{yanUnderstandingPerformanceWebassembly2021,titzerFastInplaceInterpreter2022, Daniel2019DiscoveringVI}
    中间编译表示规范\cite{wasmCommunityGroup, lehmannWasabiFrameworkDynamically2019, lehmannEverythingOldNew, 
    bhansaliFirstLookCode2022, waseemIssuesTheirCauses2024}(或称字节码)。
    wasm 字节码通过如 Emscripten 和 rustc 等 wasm 前端编译器来生成\cite{romanoEmpiricalStudyBugs2021},
    wasm 字节码格式紧凑,空间占用小,也因此用于加速处理计算密集型任务,在边缘计算和物联网领域得到了应用
    \cite{zhangResearchWebAssemblyRuntimes2024}。
    因这种字节码并不能直接交由处理器译码执行,还需经过解释器或 JIT 或 AoT(Ahead of Time)等后端编译器转译为
    与当前运行环境下CPU指令集一致的机器码\cite[6]{zhangResearchWebAssemblyRuntimes2024},wasm 有
    跨 OS 和处理器架构的可移植性和可扩展性\cite{lehmannEverythingOldNew, waseemIssuesTheirCauses2024, 
        lehmannWasabiFrameworkDynamically2019, JayProvablySafe,  zhangResearchWebAssemblyRuntimes2024,
        WebAssemblySummaryOnSecurity, rayOverviewWebAssemblyIoT2023}。

    \subsubsection{Web Assembly 内存模型}
        wasm 的内存模型定义了 wasm 程序的线性内存布局和内存对象。分配给 wasm 的内存页大小总是 64K 的倍数\cite{Daniel2019DiscoveringVI}。
        为了防止 wasm 程序越界访存,runtime需要对访存行为做边界检查。
        除此通过沙盒环境来隔离程序的安全机制以外,wasm 还通过自身的内存布局来约束程序的访存操作。
        这一线性内存并不存储全局变量或者局部变量,全局变量保存在一张固定大小的、
        名为全局索引空间(Global index space)的表上\cite{rayOverviewWebAssemblyIoT2023, 
            webassemblycommunitygroupWebAssemblySpecification2024, Daniel2019DiscoveringVI};
        局部变量存储于一个受到保护的调用栈上,调用栈保存有函数的返回地址。
        而在wasm中,线性内存既不可执行,也不可跳转,然而却不允许内存被标记为只读,所有线性内存上的数据均可写
        \cite{lehmannEverythingOldNew}。
        操作系统在执行程序时还会运用 ASLR 来随机编排栈、堆区、代码段的地址空间,但 wasm 只有线性内存并且内存编排方式一定
        \cite{lehmannEverythingOldNew}。

        但由于栈上不存在类似于金丝雀的栈溢出检查标记,wasm 有栈溢出风险。

    \subsubsection{Web Assembly 数据类型}
        wasm 借鉴了安全语言设计,采用静态强类型系统,只提供四种基本数据类型
        \cite{wasmCommunityGroup,lehmannEverythingOldNew, groupWebAssemblySpecification2024},
        分别为 32 位整数、32 位浮点数、64 位整数和 64 位浮点数。
        高级语言中的复杂类型,都在编译阶段被编译为这四类基本数据类型。
    \subsubsection{Web Assembly 运行环境}
        wasm 字节码以模块为单位,被限制于沙盒环境中加载和运行
        \cite{johnsonWaVeVerifiablySecure2023,WasmbpfStreamliningEBPF2024,haasBringingWebSpeed2017}。
        wasm 的运行时可分为 4 个组件
        \cite{zhangResearchWebAssemblyRuntimes2024, zhangCharacterizingDetectingWebAssembly2024}:
        \begin{itemize}
            \item[(1)]  wasm 解释器;
            \item[(2)]  wasm 后端编译器;
            \item[(3)]  wasm 运行时环境;
            \item[(4)]  wasi。
        \end{itemize}
        而也有部分 wasm 解释器,如 wasmer\footnote[1]{\url{https://github.com/wasmerio/wasmer}.} 和
        wasmtime\footnote[2]{\url{https://github.com/bytecodealliance/wasmtime}.},
        实现了后端编译器对wasm字节码的转译\cite{zhangResearchWebAssemblyRuntimes2024}。

        wasm 运行时环境
        这确保了其运行时安全,并让其成功地于浏览器之外为非网页程序提供沙盒环境
        \cite{narayanSwivelHardeningWebAssembly, WebAssemblySummaryOnSecurity}。
    \subsubsection{Web Assembly 存在的问题}
        截至 2022 年,贺宁宇等人在文献\parencite{caoWASMixerBinaryObfuscation2023}指出某些 Photoshop、工业 CAD 和 3D 游戏借助 wasm 而运行在浏览器上。
