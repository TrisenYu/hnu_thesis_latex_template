\section{相关工作}
\label{section:relatedWork}
	\subsection{高级编程语言}
		% https://github.com/openjdk/jdk/blob/master/src/hotspot/share/compiler/compileTask.cpp
		\subsubsection{Java字节码解释器}
		Java虚拟机(JVM)转译和执行由Scala,Java,Groovy和Kotlin等高级编程语言编译得到的Java字节码。
		且当前的JVM系统由不同的组织和公司所开发,如Oracle开发的HotSpot,
		阿里巴巴开发的DragonWell和IBM开发的OpenJ9。

		因JVM功能性复杂且规模巨大\cite{sonoyamaPerformanceStudyKotlin2021},所以潜在的缺陷不可避免\cite{jiaDetectingJVMJIT2023}。
		Haoxiang Jia et al.\cite{jiaDetectingJVMJIT2023}观察到HotSpot公开的缺陷报告在每年的总数上虽然下降,但其即时编译器部分出现的缺陷却显著增加,
		有的甚至可以引起JVM的崩溃。

		Chaliasos,et al.\cite{chaliasosWelltypedProgramsCan2021}通过按照表 \ref{jvmBugTracker} 所示的缺陷追踪网站,
		从中收集并整理出 4153 个已被修复且由测试用例确保不会再重现的缺陷;其工作还随机挑选了320个缺陷,结合开发者讨论以及缺陷报告,
		总结形成了五类的缺陷以及其根因,结果如图 \ref{JVMbugsRootCause} 所示。

		{\wuhao{
			\begin{longtable}{cccccccc}
			\caption{JVM类型缺陷的收集结果}
			\label{jvmBugTracker}\\
			\specialrule{1pt}{0pt}{0pt}
			语言 &追踪网站 &问题总数 &最早记录时间 &最近记录时间 &缺陷收集 &后-筛选量(post-filtering)\\
			\hline
			Java   &\href{https://bugs.openjdk.java.net/rest/api/latest/search}{Jira} 	&10872 	&2004-02-11 &2021-03-26 &1252 &873  \\
			Scala2 &\href{https://api.github.com/repos/scala/bug}{GitHub}				&12315 	&2003-05-22 &2021-03-11 &1180 &1067  \\
			Scala3 &\href{https://api.github.com/repos/lampepfl/dotty}{GitHub} 			&4286 	&2014-02-01 &2021-03-21 &429  &366 \\
			Kotlin &\href{https://youtrack.jetbrains.com/api/issues}{YouTrack} 			&40998 	&2011-10-28 &2021-04-09 &2189 &1601 \\
			Groovy &\href{https://issues.apache.org/jira/rest/api/2/search}{Jira} 		&9710 	&2003-09-25 &2021-04-09 &300  &246 \\
			\specialrule{1pt}{0pt}{0pt}
			\end{longtable}
		}}

		\begin{figure}[htb]
			\centering
			\pgfplotstableread{
					index	         aa ee  cc dd    bb
				类型相关错误           90 4  10  4     21
				语义分析有误           22 2  30  3     20
				解析异常              42 1  13  7     14
				错误控制与报告不当      0  0  0   7     15
				AST转换不当            5  1  0   0     9
			}\jvmBugData
			\begin{tikzpicture}
				\begin{axis}
					[
						xbar,
						xbar stacked,
						ytick=data,
						bar width=0.5cm,
						ytick style={draw=none},
						width=0.5\linewidth,
						xmin=0,
						bar shift=0pt,
						axis y line*= none, axis x line*= none,
						yticklabels from table={\jvmBugData}{index},
						legend pos=outer north east,
						legend style={column sep=8pt,},
						point meta=x,
					]
					\addlegendimage{empty legend}\addlegendentry{\xiaowu\bf{JVM症状}}
					\addplot+ table [x=aa, meta=index, y expr=\coordindex] {\jvmBugData};\addlegendentry{\xiaowu{非预期编译时错误}}
					\addplot+ table [x=ee, meta=index, y expr=\coordindex] {\jvmBugData};\addlegendentry{\xiaowu{内部编译错误}}
					\addplot+ table [x=cc, meta=index, y expr=\coordindex] {\jvmBugData};\addlegendentry{\xiaowu{非预期运行时的表现}}
					\addplot+ table [x=dd, meta=index, y expr=\coordindex] {\jvmBugData};\addlegendentry{\xiaowu{误报}}
					\addplot+[
						nodes near coords,
						every node near coord/.append style={
							xshift=24pt,
							black,
						},
					] table [x=bb, meta=index, y expr=\coordindex] {\jvmBugData};\addlegendentry{\xiaowu{编译性能问题}}
				\end{axis}
			\end{tikzpicture}
			\caption{JVM各症状的错误原因分布}
			\label{JVMbugsRootCause}
		\end{figure}

		其中,AST转换不当是指编译器没有产生和原始输入程序在逻辑上保持一致的转换程序,占JVM缺陷的4.69\%,会引发非预期的编译时错误和内部编译错误。
		错误控制与报告不当是指JVM解释器正确辨识出程序错误,但错误处理和报告机制没有产生预期的响应结果。错误控制与报告不当占JVM缺陷的6.88\%,
		此类缺陷都与崩溃和错误报告有关。
		
		解析异常则是既不能解析一个标识符(即变量名、方法名或类名),也不能正确地检索到先前解析过的标识名。解析异常也占24.06\%,由两项原因所致:
		\begin{itemize}
			\item [(1)] 解析算法执行错误,如\href{https://bugs.openjdk.java.net/browse/JDK-7042566}{JDK-7042566}是在相同函数名下选择参数更多的一个来执行,
						从而引发错误;
			\item [(2)] 编译器错误查询;
			\item [(3)] 作用域状态错误。
		\end{itemize}
		
		语义分析有误则指的是解释器对特定程序代码生成了错误的分析结果,占JVM缺陷的24.06\%,由两方面原因所致:
		\begin{itemize}
			\item [(1)] 没有验证检查语义,如问题\href{https://github.com/scala/bug/issues/5878}{Scala2-5878};
			\item [(2)] 不正确的分析机制,如问题\href{https://github.com/scala/scala3/pull/4487}{Scala3-4487}。 
		\end{itemize}

		最后的类型相关错误指的是在前端编译器中运用类型系统内定义的规则以及类型中定义的操作应用于输入的源文件之中,但因实现不当而导致的错误。
		这一部分的实现不当是造成非预期编译时错误的主要原因,且以40.31\%(129/320)的比例使之成为JVM最常见的缺陷。且细分类型相关错误则仍有三组:
		\begin{itemize}
			\item [(1)] 不正确的类型推断和类型变量代替,如缺陷报告所报告的\href{https://youtrack.jetbrains.com/issue/KT-10711}{KT-10711};
			\item [(2)] 不正确的类型(强制或非强制)转换,如缺陷报告所报告的\href{https://youtrack.jetbrains.com/issue/KT-9630}{KT-9630};
			\item [(3)] 不正确的类型比较和边界计算,如缺陷报告所报告的\href{https://bugs.openjdk.org/browse/JDK-8284879}{JDK-8039214}。
		\end{itemize}

		\subsubsection{JavaScript字节码解释器}
		JavaScript(JS)是被广泛使用的编程语言之一,其迅速的发展由超过200万个可重用的npm软件库生态所支撑\cite{bhuiyanSecBenchjsExecutableSecurity2023}。
		JS常作为客户端一侧的web应用使用,驱动了当今世界上超过90\%的web页面\cite{Staicu2018SYNODEUA}。
		这种广泛的应用使得保证JS的安全性至关重要\cite{eomFuzzingJavaScriptInterpreters2024}。

		JS通过自身的执行引擎解释执行且主流浏览器提供商均实现并管理着其编写的js引擎。Google浏览器使用V8引擎,
		Mozilla的火狐浏览器使用SpiderMonkey引擎,苹果公司的Safari浏览器使用WebKit中的JS引擎,
		微软公司开发的Edge使用ChakraCore引擎。这些引擎是浏览器的核心\cite{limSOKAnalysisWeb2021},依次将JS源码转变为字节码和机器码。

		而Ziyuan Wang et al.\cite{JSEngineBugs}汇总分析了V8,SpiderMonkey和Chakra三个主流JS引擎截至2022年
		已有的19019篇缺陷(bugs)报告,16437次修正和805个测试用例以及随机挑选的540个缺陷的根因,发现:
		\begin{itemize}
			\item[(1)] Compiler和DOM是存在缺陷最多的部件,结果如表 \ref{JSBuggySrc} 和表 \ref{V8BuggyComponents} 和表 \ref{MozillaBuggyComponents} 所示;
			\item[(2)] 小型测试用例足以触发多数漏洞,如V8中70.79\%的触发漏洞测试用例代码行数小于10行;
			\item[(3)] JS引擎漏洞修复通常不复杂,多数仅需修改少量代码行和文件。如V8中76.86\%的漏洞修复涉及代码行数少于100行;
			\item[(4)] 对于修复缺陷的时间,V8中有80.33\%的缺陷在半年内修复;SpiderMonkey引擎中有91.9\%的缺陷是在半年内修复,且有4.33\%是需要超过一年的时间才能修复;
			\item[(5)] 规范语义缺陷是最常见的根因。除此之外,执行缺陷、特性以及函数调用缺陷均比其它类型的缺陷多。与V8和Chakra相比,SpiderMonkey有更多构建错误和更少内存错误。
		\end{itemize}

		研究为确保后续研究的可复现性,还将其成果分别以BSD,MPL,MIT许可证颁布公开于Github上\footnote{\url{https://github.com/njuptw/Empirical-Study-V8}.}${}^{,}$\footnote{\url{https://github.com/njuptw/Empirical-Study-Chakra}.}${}^{,}$\footnote{\url{https://github.com/njuptw/Empirical-Study-SpiderMonkey}}。

		{\wuhao{
			\begin{longtable}{@{}@{\extracolsep{\fill}}llllll@{}}
			\caption{2022年JS引擎缺陷在源码文件中的分布}
			\label{JSBuggySrc}\\
			\specialrule{1pt}{1pt}{1pt}
			\multicolumn{2}{c}{V8} &\multicolumn{2}{c}{SpiderMonkey} &\multicolumn{2}{c}{Chakra}\\
			\hline
			源文件 &缺陷数量 &源码 &缺陷数量 &源码 &缺陷数量\\
			\hline
			\endfirsthead
			\specialrule{1pt}{1pt}{1pt}	
			\endlastfoot
			src/compiler  		&10525 		&js 			&3209 		&lib/Runtime/Library 			&4629 \\
			src/wasm  			&5070  		&dom 			&3082 		&lib/Runtime/Language 			&1984 \\
			src/builtins  		&4726  		&browser 		&2129 		&lib/Runtime/Base 				&1170 \\
			src/heap  			&3630  		&toolkit 		&1956 		&lib/Common/Memory 				&1004 \\
			src/objects  		&3087  		&third\_party 	&1933 		&lib/Runtime/ByteCode 			&1001 \\
			src/runtime  		&2571  		&gfx 			&1807 		&lib/Runtime/Types 				&883 \\
			src/parsing  		&2147  		&devtools 		&1136 		&lib/wabt/src 					&676 \\
			src/interpreter 	&1779  		&taskcluster 	&1022 		&lib/Runtime/Debug 				&582 \\
			src/crankshaft  	&1178  		&layout 		&1013 		&lib/WasmReader 				&514 \\
			src/ast  			&1141  		&intl 			&962  		&lib/Jsrt 						&482 \\
			src/full-codegen	&1089  		&netwerk 		&926  		&lib/Backend/arm64 				&264 \\
			src/debug  			&1073  		&modules 		&910  		&lib/Common/DataStructures 		&242 \\
			src/js  			&1049  		&js/src/jit 	&898  		&lib/common 					&240 \\
			src/ic  			&1034  		&mobile 		&730  		&lib/Common/Core 				&238 \\
			src/snapshot  		&1029  		&security 		&699  		&lib/Runtime/PlatformAgnostic 	&219 \\
			src/objects.cc 		&951   		&widget 		&540  		&lib/Backend/Lower.cpp 			&211 \\
			src/regexp  		&911   		&dom/media 		&534  		&lib/Backend/amd64 				&187 \\
			src/arm64  			&872   		&dom/base 		&468  		&lib/Parser/Parse.cpp 			&184 \\
			src/ia32  			&839   		&dom/ipc 		&367  		&lib/Backend/arm 				&170 \\
			src/arm  			&811   		&docshell 		&364  		&lib/Common/ConfigFlagsList.h 	&168\\
			\end{longtable}
		}}

		另外,S. Park et al.\cite{parkEmpiricalStudyPrioritizing2019}围绕JS引擎的漏洞问题,实现了CRScope用于分类安全与非安全的缺陷,
		并评估了Exploitable和AddressSanitizer等工具和模型的性能。其工作所用的训练数据集从Google、Mozilla、Microsoft的Edge与IE浏览器以及GitHub库中收集。
		内容上包含165个触发安全缺陷的JS Poc以及174个触发非安全缺陷的PoC。同时,S. Park et al.还编译了与PoC对应的JS引擎,
		构成727个二进制文件以及766个实例,用于训练模型判断崩溃问题是否由安全缺陷所引发。
		其数据集公开\footnote{\url{https://github.com/WSP-LAB/CRScope}.}以支撑后续研究。
		
		其训练模型和测试模型的方式采用了四折时间序列交叉验证方式,
		将所有安全和非安全漏洞按其在目标二进制文件中的提交日期排序,而后按升序排列崩溃实例并均分成5个桶,以保证每个桶中的崩溃实例数量大致相同(约153个)。
		评估实验结果显示,S. Park et al.所实现的CRScope则在Chakra、V8 和SpiderMonkey的崩溃实例上以0.85、0.89和0.93的准确度,
		优于Exniffer的0.51和AddressSanitizer的0.63。
		
		{\wuhao{
		\begin{longtable}{@{}@{\extracolsep{\fill}}llll@{}}
			\caption{Google开发的V8}
			\label{V8BuggyComponents}\\
			\specialrule{1pt}{0pt}{0pt}
			\multicolumn{2}{c}{未修复缺陷(5191)} & \multicolumn{2}{c}{已修复缺陷(2827)} \\
			\hline
			组件名 & 数量 &组件名称 &数量\\
			\hline
			编译器 & 932 & 编译器 & 565 \\
			WebAssembly & 855 & JS语法 & 553 \\
			JS语法 & 831 & Webassembly & 518 \\
			运行时 & 648 & 运行时 & 317 \\
			基础架构 & 446 & 垃圾回收 & 232 \\
			垃圾回收 & 435 & 人类语言国际化 & 227 \\
			人类语言国际化 & 338 & 基础架构 & 182 \\
			API & 253 & API & 111 \\
			构建文件 & 170 & 正则匹配 & 82 \\
			正则匹配 & 166 & 解释器 & 65\\
			\specialrule{1pt}{0pt}{0pt}
		\end{longtable}
		\begin{longtable}{@{}@{\extracolsep{\fill}}llll@{}}
			\caption{Mozilla开发的SpiderMonkey}
			\label{MozillaBuggyComponents}\\
			\specialrule{1pt}{0pt}{0pt}
			\multicolumn{2}{c}{未修复缺陷(4499)} & \multicolumn{2}{c}{已修复缺陷(3502)} \\
			\hline
			组件名 & 数量 &组件名称 &数量\\
			\hline
			DOM &519 &DOM &428  \\
			常规 &397 &常规 &330  \\
			JavaScript &294 &JavaScript &252  \\
			图形 &251 &图形 &211  \\
			自动化发布 &152 &自动化发布 &143  \\
			页面布局 &149 &Web平台测试 &138  \\
			Web平台测试 &144 &页面布局 &127  \\
			网络 &142 &网络 &101  \\
			音视频 1&14 &WebRTC &88  \\
			WebRTC &107 &音视频 &79 \\
			\specialrule{1pt}{0pt}{0pt}
		\end{longtable}
		}}

		\subsubsection{Lua字节码解释器}
		Lua是一个基于栈虚拟机的解释型语言,支持动态类型\cite{ierusalimschyLua54Reference2020}。
		Lua被设计为嵌入式语言,主要用于扩展应用程序,而非独立开发程序。它高度依赖宿主程序提供的API,
		标准库函数的加载也需由宿主程序显式完成\cite{SecurityLuaSandbox2022}。
		Lua官方在其官网有单独开设一个页面用于记录其解释器存在的缺陷\footnote{\url{https://www.lua.org/bugs.html}.}。
		同时还提供了各个版本的手册\footnote{\url{https://www.lua.org/manual/}.}以方便编程人员查阅和学习。
		Lua高度依赖宿主程序提供的API,甚至标准库函数的加载也需由宿主程序显式完成\cite{ierusalimschyProgrammingLua532016}。