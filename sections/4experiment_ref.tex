\section{书写规范与打印要求}
\label{section:criterions}
    \subsection{文字和字数}
    一般用汉语简化文字书写,字数在1.2万字左右,报告(内容)或软件说明书,字数在1万字左右。
    \subsection{书写}
    论文一律由本人在计算机上输入、编排并打印在A4幅面白纸上。毕业论文前置部分(即正文之前)一律用单面印刷,
    正文部分开始双面印刷。
    致谢和附录部分应单面起页双面印刷(如正文结束页为单页,则单数页背面不加页眉和页码,致谢单面起页,如致谢为单页,其背面亦不加页眉和页码)。
    \subsection{字体和字号}
    \begin{tabular}{p{7em}l}
        论文题目:   &{\hei\xiaoer{小2号黑体}}\\
        章  标  题: &{\hei\sanhao{3号黑体}}\\
        节  标  题: &{\hei\xiaosi{小4号黑体}}\\
        条  标  题: &{\hei\xiaosi{小4号黑体}}\\
        正      文: &{\song\xiaosi{小4号宋体}}\\
        页      码: &{\song\wuhao{5号宋体}}\\
        数字和字母:  &Times New Roman
    \end{tabular}


    \subsection{封面}
    论文封面规范见(样张1),论文封皮一律采用白色铜版纸,封皮大小为A4规格。
    \subsection{论文页面设置}
    \subsubsection{页眉和页脚}
        毕业论文各页均加页眉,在版心上边线隔一行1.5磅加粗 、细双线(粗线在上),其上居中打印页眉。

        页脚处居中插入页码,如 “1”。
    \subsubsection{页边距}
        上边距:30mm;下边距:25mm;左边距:30mm;右边距:20mm;行间距为1.5倍行距。
    \subsubsection{页码的书写要求}
        论文页码从绪论部分开始,至附录,用阿拉伯数字连续编排,页码位于页脚居中。封面、摘要和目录不编入论文页码;摘要和目录用罗马数字单独编页码。
    
    \subsection{摘要}
    \subsubsection{中文摘要}
        中文摘要包括:论文题目(小3号黑体)、“摘要”字样(3号黑体)、摘要正文(小4号宋体)和关键词。

        摘要正文后下空一行打印“关键词”三字(4号黑体),关键词(小4号黑体)一般为3~5个,每一关键词之间用分号分开,
        最后一个关键词后不打标点符号,见(样张3)。

    \subsubsection{英文摘要}
        英文摘要另起一页,其内容及关键词应与中文摘要一致,并要符合英语语法,语句通顺,文字流畅。

        英文题目的字样为:The title(小3号 Times New Roman 加粗)

        英文摘要字样为:Abstract(3号 Times New Roman 加粗)

        然后隔行书写摘要的文字部分。(字体为小4号Times New Roman)

        摘要正文后下空一行打印关键词(4号Times New Roman加粗):key word1;key word2;(关键词3-5个,小4号Times New Roman加粗)

        英文和汉语拼音一律为Times New Roman体,字号与中文摘要相同,见(样张4)。
    \subsection{目录}
    专业目录的三级标题,建议按(1……、1.1……、1.1.1……)的格式编写,目录中各章题序的阿拉伯数字用Times New Roman体,第一级标题用小4号黑体,其余用小4号宋体。目录的打印实例见(样张5)。

    
    \subsection{论文正文}
    \subsubsection{章节及各级标题}
        论文正文分章节撰写, 每章应另起一页。各章标题要突出重点、简明扼要。字数一般在15字以内, 不得使用标点符号。
        标题中尽量不采用英文缩写词,对必须采用者,应使用本行业的通用缩写词。

    \subsubsection{层次}
        层次以少为宜,根据实际需要选择。正文层次的编排和代号要求统一,层次为章(如“1”)、节(如“1.1”)、条(如“1.1.1”)、款(如“1.”)、
        项(如“(1)”)。层次用到哪一层次视需要而定,
        若节后无需“条”时可直接列“款”、“项”。“节”、“条”的段前、段后各设为0.5行,见(样张6)。

    \subsection{引用文献}
    引用文献标示方式应全文统一,并采用所在学科领域内通用的方式,
    所引文献编号用阿拉伯数字置于方括号中,用上标的形式置于所引内容最末句的右上角,如:“…成果[1]” ,
    引用文献应与文中标注一致。几处地方引用同一个文献时,文中标注按第一次出现的序号。
    当提及的参考文献为文中直接说明时,其序号应该用阿拉伯数字与正文排齐,如“由文献[8, 10-14]可知”。

    不得将引用文献标示置于各级标题处。
    \subsection{名词术语}
    科技名词术语及设备、元件的名称,应采用国家标准或部颁标准中规定的术语或名称。标准中未规定的术语要采用行业通用术语或名称。
    
    全文名词术语必须统一。一些特殊名词或新名词应在适当位置加以说明或注解。采用英语缩写词时, 除本行业广泛应用的通用缩写词外, 
    文中第一次出现的缩写词应该用括号注明英文全文。如返回导向编程(Return Oriented Programming, ROP)。
    \subsection{物理量名称符号及计量单位}
    \subsubsection{物理量的名称和符号}
        物理量的名称和符号应符合GB3100~3102-86的规定。论文中某一量的名称和符号应统一。
    \subsubsection{物理量计量单位}
        物理量计量单位及符号应按国务院1984年发布的《中华人民共和国法定计量单位》及GB3100~3102执行, 不得使用非法定计量单位及符号。计量单位符号,除用人名命名的单位第一个字母用大写之外,其它一律用小写字母。

        非物理量单位(如件、台、人、元、次等)可以采用汉字与单位符号混写的方式,如“万t·km”。

        文稿叙述中不定数字之后允许用中文计量单位符号, 如“几千克至1000kg”。

        表达时刻时应采用中文计量单位,如“上午8点3刻”,不能写成“8h45min”。

        计量单位符号一律用正体。

    \subsection{正体斜体用法规定}
    物理量符号、物理常量、变量符号用{\bfseries\song{斜体}},计量单位等符号均用{\bfseries\song{正体}},见(样张7(1))。
    \subsection{数字}
    除习惯用中文数字表示的以外, 一般均采用阿拉伯数字。年份一概写全数,如2003年不能写成03年。
    \subsection{公式}
    公式应另起一行写在稿纸中央,公式和编号之间不加虚线。公式较长时最好在等号“=”处转行。
    如难实现,则可在$+$、$-$、$\times$、$\div$运算符号处转行,运算符号应写在转行后的行首,公式的编号用圆括号括起来放在公式右边行末。

    公式序号按章编排,如第一章第一个公式序号为“(1.1)”, 附录A中的第一个公式为“(A1)”等。

    文中引用公式时,一般用“见式(1.1)”或“由公式(1.1)”。
    公式中用斜线表示“除”的关系时应采用括号, 以免含糊不清, 如$a/(b\cos x)$。通常“乘”的关系在前,如$a\cos x/b$而不写成$(a/b)\cos x$。
